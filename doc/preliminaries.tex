\newcommand\aut{\mathcal{A}}
\newcommand\sem[1]{[\![ #1 ]\!]}

\newcommand\cfarightarrow[1]{\xrightarrow{#1}}
\newcommand\vln{\theta}
\newcommand\Vlns{\Theta}
\newcommand\op{\red{op}}
\newcommand\Ops{\red{Ops}}
\newcommand\states{S}
\newcommand\labels{L}
\newcommand\trans{R}
\newcommand\exect{\textsf{exec}}
\newcommand\rvs{Y}%\overline{r}}
\newcommand\exec[2]{\exect(\vln_{#1},m,\vln_{#2},\rvs)}
\newcommand\execVVM[3]{\exect(\vln_{#1},#3,\vln_{#2},\rvs)}
\newcommand\qv[1]{\langle q_{#1}, \vln_{#1} \rangle}
\newcommand\qtv[1]{\langle q_{#1}, `\vln_{#1} \rangle}

\section{Preliminaries}

\red{cfas or asts?}

\begin{definition}[Control-flow automaton~\cite{Henzinger2002}]
A (deterministic) \emph{control flow automaton} denoted $\aut$, is a
tuple $\aut = \langle Q, q_0, X, s, \cfarightarrow{\,} \rangle$ where $Q$ is a
finite set of control locations and $q_0$ is the initial control location, $X$
is a finite sets of typed variables, $s$ is the statement language (as defined
earlier) and $\cfarightarrow{\,}\subseteq Q \times s \times Q$ is a finite set
of labeled edges.
\end{definition}

\newcommand\run{r}

\paragraph{Valuations and semantics.}
We define a \emph{valuation} of variables 
$\vln : X \rightarrow \mathit{Val}$ to be a mapping from
variable names to values. Let $\Vlns$ be the set of all valuations.
%
The notation $\vln' \in \sem{s}\vln$ means that executing statement $s$, using
the values given in $\vln$, leads to a new
valuation $\vln'$, mapping variables $X$ to new values. Notations $\sem{e}\vln$ and $\sem{b}\vln$ represent side-effect free numeric and boolean valuations, respectively.
%
We assume that for every $\vln,s$, that $\sem{s}\vln$ is computed in finite time.
%
The automaton edges, along with $\sem{s}$, give rise to possible transitions, denoted
$(q,\vln) \cfarightarrow{s} (q',\vln')$ (we omit these rules for lack of space.

A \emph{run} of a CFA is an alternation of automaton states and
valuations: $\run=q_0,\vln_0,q_1,\vln_1,q_2,\dots$
such that $\forall i \geq0.\ (q_i,\vln_i) \cfarightarrow{s} (q_{i+1},\vln_{i+1})$.
%\paragraph{Safety and termination.}
%
We say that CFA $\aut$ can \emph{reach} automaton state $q$ (\emph{safety}) provided that
there exists a run $\run= q_0,\vln_0,q_1,\vln_1,...$ such that there is
some $i \geq 0 $ such that  $q_i=q$.
%
%We say that CFA $\aut$ \emph{terminates} provided that there
%are no infinite runs.
%$\forall q_0,\vln_0,q_1,\vln_1,...\in\Pi(\aut)$ such that there is
%some $i \geq 0 $ such that  $q_i=q$.